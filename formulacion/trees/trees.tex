\section{Arbol de Problemas}

\begin{figure}[H]
    \centering
    \includegraphics[width=0.9\linewidth]{formulacion/trees/img/arbol_problemas_APS.png}
    \caption{Arbol de Problemas.}
    \label{fig:placeholder}
    \vspace{0.2cm}
    \small Fuente: Elaboración propia.
\end{figure}

\subsection{Fundamentos del Árbol de Problemas}
\begin{itemize}

    \item \textbf{Poca integración locativa} \\
    En Colombia, la falta de articulación entre los organismos de tránsito municipales y departamentales y la infraestructura urbana ha generado congestión y sobrecarga en los puntos de atención. El crecimiento sostenido del parque automotor en el país ha superado la capacidad física de muchas sedes de tránsito, especialmente en ciudades intermedias, lo que provoca saturación en la atención presencial y congestión vial en zonas cercanas a estas oficinas.  
    \textbf{(Ministerio de Transporte, 2024; DANE, 2024)}

    \item \textbf{Poca productividad de los funcionarios} \\
    A nivel nacional, los trámites para obtener o renovar licencias pueden tardar varias semanas dependiendo del municipio. Aunque existe el sistema centralizado del RUNT, muchos procesos requieren validaciones manuales y múltiples pasos presenciales. La persistencia de procedimientos burocráticos y la dependencia de otras entidades como centros médicos y escuelas de conducción reduce la eficiencia operativa.  
    \textbf{(RUNT, 2024; Función Pública, 2023)}

    \item \textbf{Falta de personal de apoyo} \\
    En diversas regiones del país, los organismos de tránsito presentan déficit de agentes y personal administrativo, lo que limita la capacidad para atender trámites y ejercer control vial eficiente. El crecimiento del parque automotor no ha sido proporcional al aumento del talento humano en las entidades de tránsito, generando sobrecarga laboral y retrasos en la gestión.  
    \textbf{(Ministerio de Transporte, 2024; Contraloría General de la República, 2023)}

    \item \textbf{Poca infraestructura tecnológica} \\
    Aunque el Gobierno Nacional ha promovido procesos de digitalización, muchas oficinas de tránsito presentan sistemas tecnológicos con baja interoperabilidad y dependencia de servidores centrales. Las fallas en plataformas digitales afectan la continuidad de los trámites en diferentes regiones del país.  
    \textbf{(RUNT, 2024; Ministerio TIC, 2024)}

    \item \textbf{Procedimientos manuales} \\
    En múltiples regiones de Colombia aún se exige la entrega física de documentos, validaciones presenciales y revisión manual de requisitos. Estos procedimientos ralentizan la atención, incrementan el margen de error humano y generan reprocesos innecesarios.  
    \textbf{(Función Pública, 2023; Consejo Privado de Competitividad, 2024)}

    \item \textbf{Problemática principal: Trámites para licencias lentos y tediosos} \\
    En Colombia, la gestión de licencias de conducción presenta demoras constantes y procesos extensos. En distintas ciudades se reportan largas filas para renovar o expedir la licencia. Además, las fallas técnicas y mantenimientos del RUNT han paralizado temporalmente trámites como expediciones, renovaciones y recategorizaciones, evidenciando una alta dependencia tecnológica y limitada capacidad operativa.  
    \textbf{(RUNT, 2024; Ministerio de Transporte, 2024)}

    \item \textbf{Aumento de los reprocesos para los usuarios} \\
    Las fallas en el RUNT y la baja interoperabilidad entre entidades generan reprocesos frecuentes. Cuando la plataforma presenta fallas, los usuarios deben reagendar citas, volver a presentar documentos o repetir exámenes médicos, aumentando los costos económicos y emocionales.  
    \textbf{(RUNT, 2024; Función Pública, 2023)}

    \item \textbf{Pérdida productiva para los ciudadanos} \\
    El tiempo invertido en trámites administrativos reduce la productividad laboral. En Colombia, las empresas destinan miles de horas al año a procesos burocráticos, lo que impacta especialmente a trabajadores independientes y microempresarios que dependen del vehículo como herramienta de trabajo.  
    \textbf{(Consejo Privado de Competitividad, 2024; DANE, 2024)}

    \item \textbf{Baja del PIB nacional} \\
    La acumulación de ineficiencias administrativas impacta negativamente la competitividad del país. Los retrasos en trámites relacionados con movilidad afectan sectores estratégicos como transporte de carga, logística e infraestructura, reduciendo el impulso productivo nacional.  
    \textbf{(Consejo Privado de Competitividad, 2024; DANE, 2024)}

    \item \textbf{Aumento de costos para el Estado} \\
    La ineficiencia burocrática incrementa el gasto público. La necesidad de mantener sistemas tecnológicos desactualizados y reforzar personal operativo genera costos adicionales no previstos, reduciendo recursos disponibles para inversión en modernización e infraestructura.  
    \textbf{(Contraloría General de la República, 2023; Función Pública, 2023)}

    \item \textbf{Disminución del desempeño de inversión para el desarrollo} \\
    Los retrasos administrativos afectan proyectos de infraestructura y movilidad en Colombia. Procesos lentos generan sobrecostos, retrasos contractuales y menor confianza institucional, afectando la dinámica de inversión y desarrollo nacional.  
    \textbf{(Consejo Privado de Competitividad, 2024; Ministerio de Transporte, 2024)}

\end{itemize}


\section{Arbol de Soluciones}

\begin{figure}[H]
    \centering
    \includegraphics[width=1\linewidth]{formulacion/trees/img/Arbol_Soluciones_APS.png}
    \caption{Arbol de Soluciones.}
    \label{fig:placeholder}
     \vspace{0.2cm}
    \small Fuente: Elaboración propia.
\end{figure}

\section{Bibliografía}
\begin{itemize}

    \item Ministerio de Transporte. (2024). \textit{Estadísticas del parque automotor en Colombia}. Recuperado de: \url{https://www.mintransporte.gov.co/publicaciones/estadisticas-parque-automotor}

    \item Departamento Administrativo Nacional de Estadística (DANE). (2024). \textit{Indicadores económicos y estadísticas de transporte en Colombia}. Recuperado de: \url{https://www.dane.gov.co/index.php/estadisticas-por-tema/transporte}

    \item Registro Único Nacional de Tránsito (RUNT). (2024). \textit{Informe de operación y funcionamiento del sistema RUNT}. Recuperado de: \url{https://www.runt.com.co/estadisticas}

    \item Departamento Administrativo de la Función Pública. (2023). \textit{Informe sobre eficiencia administrativa y simplificación de trámites en Colombia}. Recuperado de: \url{https://www.funcionpublica.gov.co/web/eva/gestion-publica}

    \item Contraloría General de la República. (2023). \textit{Informe sobre eficiencia del gasto público y modernización institucional}. Recuperado de: \url{https://www.contraloria.gov.co/control-fiscal/informes}

    \item Ministerio de Tecnologías de la Información y las Comunicaciones (MinTIC). (2024). \textit{Avances en transformación digital del Estado colombiano}. Recuperado de: \url{https://www.mintic.gov.co/portal/inicio/Transformacion-Digital}

    \item Consejo Privado de Competitividad. (2024). \textit{Informe Nacional de Competitividad 2024}. Recuperado de: \url{https://compite.com.co/informe-nacional-de-competitividad}

\end{itemize}
