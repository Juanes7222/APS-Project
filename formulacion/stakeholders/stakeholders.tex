\section{Involucrados}

\begin{itemize}
	\item Usuarios
	\item Escuelas de conducción
	\item Centros de reconocimiento de conductores (CRC)
	\item Centros de enseñanza automovilística (CEA)
	\item RUNT (Registro Único Nacional de Tránsito)
	\item Ministerio de Transporte
	\item Secretarías de Tránsito 
	\item Entidades de salud (IPS)
	\item Agencias de seguros
	\item Proveedores de mensajería
	\item Gobernaciones y alcaldías
	\item Entidades financieras
	\item Tramitadores
    \item {Asociaciones y gremios de taxistas y transporte urbano}
	      \begin{itemize}
            \item APROCTAX – gremio de taxistas.
            \item FENALTAX – Federación Nacional de Taxistas.
            \item SINDINAL (Sindicato Nacional de Taxistas / Taxistas organizados).
        \end{itemize}
    \item {Asociaciones del transporte público y especial}
	      \begin{itemize}
            \item Asociación Colombiana de Minibuses (ACOLMI).
            \item Asociación Colombiana de Transporte Especial y Turismo (ACOLTES).
            \item Confederación Nacional de Transporte Especial (CONALTRAES).
            \item Cooperativa de Transportadores Escolares (COOTRAESCOLAR).
        \end{itemize}
    \item {Gremios y asociaciones de transporte de carga y camioneros}
	      \begin{itemize}
            \item Asociación Colombiana de Camioneros (ACC) – gremio de camioneros y transportadores por carretera.
            
            \item Federación Colombiana de Transportadores de Carga por Carretera (Colfecar).
            
            \item Fedetranscol – Federación de Empresas Transportadoras de Carga de Colombia.
            
            \item Asociación Nacional de Empresas Transportadoras de Carga (ASECARGA).
            
            \item Asociación de Transportadores de Carga (ATC).
            
            \item Asociación Nacional de Transportadores (ANT).
            
            \item Asociación Nacional de Transportadores Especiales y Turismo (ASONALTET).
        \end{itemize}
	\item ADITT (Asociación para el Desarrollo Integral del Transporte Terrestre Intermunicipal)
\end{itemize}


\section{Grupos de Involucrados}

\begin{itemize}
	\item \textbf{Particulares:}
	      Incluyen a las personas y actores individuales que interactúan directamente con el sistema de tránsito.
	      \begin{itemize}
		      \item \textbf{Usuarios:} Personas que requieren tramitar su licencia. Son la población directamente beneficiada y quienes hacen uso de los servicios para poder conducir de forma legal.
		      \item \textbf{Tramitadores:} Personas que intermedian en procesos de tránsito para facilitar o agilizar la gestión documental de terceros.
	      \end{itemize}

	\item \textbf{Formación y certificación:}
	      Son las instituciones que garantizan la preparación y validación de las competencias necesarias para conducir.
	      \begin{itemize}
		      \item \textbf{Escuelas de conducción:} Centros formativos que imparten enseñanza teórica y práctica para obtener permisos o licencias de conducción. Suelen contar con circuitos de práctica, vehículos y profesorado especializado (Wikipedia, s.f.).
		      \item \textbf{Centros de reconocimiento de conductores (CRC):} Entidades que realizan evaluaciones médicas, psicológicas y físicas para certificar que un aspirante está en condiciones de conducir (Centro de Reconocimiento de Conductores CRC,2023).
		      \item \textbf{Centros de enseñanza automovilística (CEA):} Organizaciones que refuerzan la formación vial mediante cursos de actualización, reentrenamiento y programas complementarios (Centro de Enseñanza Automovilística (CEA), s/f).
		      \item \textbf{Entidades de salud (IPS):} Establecimientos (públicos, privados, mixtos o comunitarios) autorizados para prestar servicios clínicos, hospitalarios y de atención médica directamente a los usuarios del POS (Ley 100 de 1993, art. 156; Corte Constitucional, s. f.; Bogotá.gov.co, 2021).
	      \end{itemize}

	\item \textbf{Estado:}
	      Entidades públicas responsables de regular, supervisar y garantizar el cumplimiento de las normas de tránsito (Colombia, s/f).
	      \begin{itemize}
		      \item \textbf{Ministerio de Transporte:} Entidad encargada de formular políticas, reglamentar y supervisar el sistema de transporte a nivel nacional (Mintransporte, 2011).
		      \item \textbf{RUNT:} Base de datos nacional que centraliza información sobre conductores, vehículos y trámites, garantizando control y trazabilidad en el sistema de tránsito (¿Qué es el RUNT?, s/f).
		      \item \textbf{Secretarías de Tránsito:} Secretaría de Tránsito de Pereira: Entidad municipal responsable de coordinar el cumplimiento del plan de tránsito, tramitar procedimientos técnicos y gestionar cultura vial en Pereira (Pereira, 2023).
		      \item \textbf{Gobernaciones y alcaldías:} La Gobernación de Risaralda es la entidad del gobierno departamental encargada de administrar y coordinar los asuntos públicos del departamento. La Alcaldía de Pereira es la entidad municipal que representa el gobierno local de la ciudad (Federación Nacional de Departamentos, s. f.).
	      \end{itemize}

	\item \textbf{Empresas:}
	      Actores del sector privado que prestan servicios complementarios para el funcionamiento del sistema de tránsito.
	      \begin{itemize}
		      \item \textbf{Agencias de seguros:} Empresas que actúan como intermediarias entre los clientes (personas o empresas) y las compañías aseguradoras. Asesoran, venden y gestionan pólizas, captan nuevos clientes y renuevan pólizas existentes (Sigo Seguros, 2024).
		      \item \textbf{Entidades financieras:} Instituciones (como bancos) que gestionan servicios financieros: administración y préstamo de dinero, manejo de cuentas, inversión y crédito (Fondo de Pensiones de Colombia [FOPEP], s. f.).
		      \item \textbf{Proveedores de mensajería:} Empresas especializadas en recoger, transportar y entregar documentos o paquetes. El servicio expreso (o Courier) opera con rapidez y cubre envíos puerta a puerta, a nivel nacional o internacional, con seguimiento y eficiencia (Inter Rapidísimo, s. f.)
		      \item \textbf{Gremios de Transporte de Carga:} Asociaciones como \textbf{ACC, Colfecar, Fedetranscol, ASECARGA, ATC, ANT y ASONALTET}, que representan a camioneros y empresas logísticas en temas de fletes y seguridad vial (Colfecar, 2024).
		      \item \textbf{Asociaciones de Transporte Público y Especial:} Organizaciones como \textbf{ACOLMI, ACOLTES, CONALTRAES y COOTRAESCOLAR}, que agrupan transportadores escolares, empresariales y de turismo (Acoltes, 2023).
		      \item \textbf{Asociaciones de Taxistas:} Entidades como \textbf{APROCTAX, FENALTAX y SINDINAL}, conformadas para defender los derechos de los conductores de taxi y promover proyectos comunes (Informacolombia, s.f.).
		      \item \textbf{ADITT:} Organización que representa a empresas y conductores del transporte intermunicipal (Conózcanos, s/f).
	      \end{itemize}
\end{itemize}
\section{Mapa de involucrados (Mentefacto)}
\begin{figure}[H]
	\centering
	\includegraphics[width=1\linewidth]{formulacion/stakeholders/img/mentefacto.png}
	\caption{Mapa de Involucrados (Mentefacto)}
	\label{fig:placeholder}
	\vspace{0.2cm}
	\small Fuente: Elaboración propia.
\end{figure}
\clearpage


\section{Matriz de Problemas - Intereses Percibidos}
\begin{table}[!ht]
	\centering
	\begin{tabular}{|p{4cm}|p{5cm}|p{6cm}|}
		\hline
		\textbf{Involucrado}     & \textbf{Interés}                                           & \textbf{Problema}                               \\ \hline
		Usuarios                 & Trámites rápidos y claros                                  & Trámites lentos y confusos.                     \\ \hline
		Escuelas de conducción   & Aumentar visibilidad y captar más estudiantes.             & Escasa visibilidad y pérdida de estudiantes. \\ \hline
		CRC                      & Incrementar agendamiento de exámenes médicos.              & Ineficiencia en agendamiento de citas médicas.        \\ \hline
		CEA                      & Agilizar inscripción de aspirantes.                        & Inscripción lenta y engorrosa de aspirantes.          \\ \hline
		RUNT                     & Registros no duplicados y alta trazabilidad.               & Registros duplicados y baja trazabilidad. \\ \hline
		Ministerio de Transporte & Mejorar digitalización y control.                          & Digitalización deficiente y no bajo control.      \\ \hline
		Secretarías de Tránsito  & Descongestionar oficinas.                                  & Congestión presencial por gestión ineficiente.         \\ \hline
		IPS certificadoras       & Optimizar atención de usuarios.                            & Saturación y demoras en atención a los usuarios.           \\ \hline
		Agencias de seguros      & Ofrecer pólizas asociadas a licencias.                     & Inexistencia de integración de oferta de pólizas para licencias.                   \\ \hline
		Mensajería               & Alta demanda de mensajería.                                & Baja demanda de mensajería por trámites presenciales.      \\ \hline
        Gobernaciones y alcaldías  & Lograr modernización tecnológica.                                & Rezago tecnológico en la gestión pública. \\ \hline
		Entidades financieras    & Facilitar pagos digitales.                                 & Pagos digitales limitados y trámites presenciales.                \\ \hline
		Tramitadores             & Mantener ingresos por gestión de trámites.                 & Reducción de ingresos por digitalización parcial.            \\ \hline
		AMCO                     & Implementar procesos ágiles.                               & Resistencia al cambio y baja capacitación.                \\ \hline
		Asociación de Taxistas   & Trámites ágiles y simples para conductores.                 & Trámites demorados y engorrosos.                \\ \hline
		Asociaciones y gremios de taxistas y transporte urbano & Trámites eficientes y de fácil acceso. & Trámites ineficientes y de difícil acceso. \\ \hline
	\end{tabular}
	\caption{Matriz de problemas — intereses percibidos (parte 1)}
    \vspace{0.2cm}
    \small Fuente: Elaboración propia.
\end{table}


\newpage
\section{Matriz de Expectativas - Fuerza}
\begin{table}[H]
\centering
\begin{tabular}{|p{4cm}|c|c|c|p{3cm}|}
\hline
\textbf{Involucrado} & \textbf{Expectativa} & \textbf{Fuerza} & \textbf{Resultado} & \textbf{Rol} \\ \hline
Usuarios & 5 & 3 & 15 & Favorecedor \\ \hline
Escuelas de conducción & 3 & 3 & 9 & Favorecedor \\ \hline
CRC & 2 & 3 & 6 & Neutral \\ \hline
CEA & 2 & 3 & 6 & Neutral \\ \hline
RUNT & 4 & 5 & 20 & Favorecedor \\ \hline
Ministerio de Transporte & 4 & 5 & 20 & Favorecedor \\ \hline
Secretarías de Tránsito & 3 & 5 & 15 & Favorecedor \\ \hline
IPS certificadoras & 2 & 3 & 6 & Neutral \\ \hline
Agencias de seguros & 1 & 2 & 2 & Neutral \\ \hline
Mensajería & 1 & 2 & 2 & Neutral \\ \hline
Gobernaciones y alcaldías & 3 & 4 & 12 & Favorecedor \\ \hline
Entidades financieras & 2 & 3 & 6 & Neutral \\ \hline
Tramitadores & -3 & 2 & -6 & Neutral \\ \hline
AMCO & 4 & 3 & 12 & Favorecedor \\ \hline
Asociación de Taxistas & 4 & 3 & 12 & Favorecedor \\ \hline
Asociaciones y gremios de taxistas y transporte urbano & 4 & 3 & 12 & Favorecedor \\ \hline
\end{tabular}
\caption{Matriz expectativa–fuerza}
\vspace{0.2cm}
\small Fuente: Elaboración propia.
\end{table}


\section{Análisis de Involucrados}
\justify
Los usuarios centran su interés en la agilización y claridad de los trámites, debido a la lentitud y confusión actual. Presentan una expectativa muy alta (5) y una fuerza intermedia (3), lo que los ubica como favorecedores del proyecto.

Las escuelas de conducción buscan mayor visibilidad y captación de estudiantes ante la pérdida actual de alumnos. Tienen una expectativa y fuerza intermedia (3–3), por lo que también cumplen un rol favorecedor.

Los CRC y los CEA están interesados en agilizar los procesos de inscripción y agendamiento, actualmente ineficientes. Ambos presentan baja expectativa (2) y fuerza intermedia (3), adoptando un rol neutral.

El RUNT y el Ministerio de Transporte enfocan su interés en mejorar la trazabilidad, evitar duplicidad de registros, fortalecer la digitalización y el control. Ambos tienen alta expectativa (4) y alta fuerza (5), posicionándose como actores clave y favorecedores estratégicos del proyecto.

Las secretarías de tránsito, así como las gobernaciones y alcaldías, buscan descongestionar oficinas y modernizar tecnológicamente la gestión pública. Presentan expectativa intermedia (3–4) y fuerza alta (4–5), lo que las convierte en favorecedoras relevantes.

Las IPS certificadoras de tránsito y las entidades financieras muestran baja expectativa (2) y fuerza intermedia (3), por lo que su rol es neutral. Las agencias de seguros y la mensajería presentan baja expectativa (1) y baja fuerza (2), manteniéndose también neutrales.

Los tramitadores informales tienen una expectativa muy baja (-5) debido a la posible disminución de sus ingresos, y cuentan con baja fuerza (1), adoptando un rol neutral con limitada incidencia.

Finalmente, el AMCO, la asociación de taxistas y la ADITT presentan alta expectativa (4) y fuerza intermedia (3), mostrando una postura favorecedora frente al desarrollo de trámites más ágiles, eficientes y accesibles.


\printbibliography

\section{Bibliografía}

\begin{itemize}
    \item RUNT (Portal Oficial) \url{https://www.runt.com.co/sobre-runt/que-es-el-runt}. Consultado el 9 de febrero de 2026.
    \item Ministerio de Transporte. Disponible en: \url{https://www.mintransporte.gov.co/}. Consultado el 9 de febrero de 2026.
    \item Colfecar (Gremio de Carga). Disponible en: \url{https://colfecar.org.co/}. Consultado el 9 de febrero de 2026.
    \item ACOLTES (Transporte Especial). Disponible en: \url{https://colfecar.org.co/}. Consultado el 9 de febrero de 2026.
    \item ADITT (Transporte Intermunicipal): \url{https://aditt.org/}. Consultado el 9 de febrero de 2026.
    \item Superintendencia de Transporte (Sobre Tramitadores): \url{https://www.supertransporte.gov.co/}. Consultado el 9 de febrero de 2026.
    \item Federación Nacional de Departamentos (Gobernaciones): \url{https://www.fnd.org.co/}. Consultado el 9 de febrero de 2026.
\end{itemize}